\documentclass[11pt,a4paper]{article}
\usepackage{minted}
\usepackage[utf8]{inputenc}
\usepackage[T1]{fontenc}
\usepackage{amsmath}
\usepackage{array, multirow, bigdelim, makecell, booktabs} 
\usepackage{color}
\usepackage{amsfonts}
\usepackage[dvipsnames,table,xcdraw]{xcolor}
\usepackage{amssymb}
\usepackage{caption}
\usepackage{subcaption}
\usepackage{wrapfig}
\usepackage{lscape}
\usepackage{rotating}
\usepackage[spanish]{babel}
\usepackage[section]{placeins}
\usepackage{subcaption}
\usepackage{graphicx}
\usepackage{epstopdf}
\usepackage[font=scriptsize,labelfont=bf]{caption}
\usepackage{anysize}
\marginsize{3cm}{3cm}{2.5cm}{2.5cm}
\providecommand{\abs}[1]{\lvert#1\rvert}
\providecommand{\norm}[1]{\lVert#1\rVert}
\renewcommand{\thesubsection}{\thesection.\alph{subsection}}
\author{Emanuel Alfredo Cortez Médici}
\title{Trabajo Práctico 1}
\DeclareGraphicsExtensions{.jpeg,.png}
\usepackage{beramono}% monospaced font with bold variant

\usepackage{listings}
\usepackage{color}

\definecolor{shellGreen}{RGB}{19,193,106}
\definecolor{backcolor}{rgb}{0.95,0.95,0.92}
\definecolor{mateBlack}{RGB}{45,45,50}
\definecolor{comment}{rgb}{0.1,0.6,0.2}
\definecolor{codegray}{rgb}{0.5,0.5,0.5}

\lstdefinestyle{vhdl}{
   language=vhdl,
   frame=single,
   basicstyle=\scriptsize,
   breaklines=true,
   captionpos=b,
   keepspaces=true,
   backgroundcolor=\color{backcolor},
   keywordstyle=[1]\color{blue}\bf,
   keywordstyle=[2]\color{red}\bf,
   keywordstyle=[3]\color{cyan!50}\bf,
   stringstyle=\color{orange},
   commentstyle=\color{comment},
   tabsize=2,
%   number=left,
%   numberstep=5pt,
   showspaces=false,
   showstringspaces=false,
   showtabs=false,
   moredelim=[is][\component]{component\ }{\ is},
   morekeywords=[1]{
      library, use ,all,entity,is,port,in,out,end,architecture,of, body,
      function, variable, begin,and,or,Not,downto,ALL, signal, process, if,
      else, elsif, case, when, then, range, to, component, type, with, select,
      others, constant, inout, buffer, map, true, false, array, subtype, wait,
      wait for, generic, =, <, >, <=, >=, =>,
   },
   alsoletter={=, <, >},
   morekeywords=[2]{
          STD_LOGIC_VECTOR,STD_LOGIC,IEEE,STD_LOGIC_1164, work, local, real,
          math_real, time, NUMERIC_STD,STD_LOGIC_ARITH,STD_LOGIC_UNSIGNED,
          std_logic_vector, std_logic, ieee, numeric_std, std_ulogic,
          std_logic_1164, natural, bit, bit_vector, signed, unsigned,
          boolean, integer
    },
    morekeywords=[3]{rising_edge, falling_edge, resize, to_signed, to_unsigned},
    morecomment=[l]{--},
    morecomment=[s][\color{orange}]{'}{'},
    rulecolor=\color{black},
}
\def\component#1{%
    \textbf{\textcolor{blue}{component\ }}%
    \textcolor{green}{#1}%
    \textbf{\textcolor{blue}{\ is}}%
}

\lstset{style=vhdl,
	breaklines=true,
    postbreak=\mbox{\textcolor{red}{$\hookrightarrow$}\space}
}





\begin{document}
	\begin{titlepage}
	\centering
	
	\includegraphics[width=0.1\textwidth]{fotos/unsl.png} 
	
	{\scshape\LARGE Universidad Nacional de San Luis\par}
	{\scshape Facultad de Ciencias Físico Matemáticas y Naturales.\par}
	{\scshape Ingeniería Electrónica con O.S.D.\par}
	\vspace{1.5cm}
	{\scshape\Large Arquitectura de computadoras \par}
	\vspace{1.5cm}
	{\huge\bfseries Trabajo Práctico N$^{\text{o}}$ 4
	
    Camino de datos y control del microprocesador\par}
	\vspace{2cm}
	Alumno\par
	{\Large\itshape 	
	Cortez Médici Emanuel\par}
	\vfill
	Profesores Responsables\par
	Ing. Andrés~\textsc{Airabella}\\
	Ing. Astrid~\textsc{Andrada}
	
	\vfill

% Bottom of the page
	{\large \today\par}
\end{titlepage}

\newpage

\section{Genere dibujos independientes para cada uno de los bloques que va a utilizar para
construir el procesador.}

Se generan los siguientes bloques independientes para 

\section{Realice el dibujo de un Datapath completo para el set de instrucciones propuesto.
Indique en el dibujo anchos de todos los buses y nombres de las señales
intermedias que luego utilizará en el código.}

\section{Realice una tabla de verdad para todas las señales de control.}

\section{Cree un repositorio en www.gitlab.com siguiendo este tutorial:
https://alejandrojs.wordpress.com/2017/06/01/como-empezar-a-usar-git-con-gitlab/}

\section{Elabore un informe con el desarrollo de los ejercicios. Este informe deberá
escribirse dentro del mismo repositorio creado en el ejercicio anterior, utilizando el
formato “Markdown”.}


\end{document}