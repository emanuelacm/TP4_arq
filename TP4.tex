\documentclass[11pt,a4paper]{article}
\usepackage{minted}
\usepackage[utf8]{inputenc}
\usepackage[T1]{fontenc}
\usepackage{amsmath}
\usepackage{array, multirow, bigdelim, makecell, booktabs} 
\usepackage{color}
\usepackage{amsfonts}
\usepackage[dvipsnames,table,xcdraw]{xcolor}
\usepackage{amssymb}
\usepackage{caption}
\usepackage{subcaption}
\usepackage{wrapfig}
\usepackage{lscape}
\usepackage{rotating}
\usepackage[spanish]{babel}
\usepackage[section]{placeins}
\usepackage{subcaption}
\usepackage{graphicx}
\usepackage{epstopdf}
\usepackage[font=scriptsize,labelfont=bf]{caption}
\usepackage{anysize}
\usepackage{tikz}
\usetikzlibrary{arrows,shapes.gates.logic.US,shapes.gates.logic.IEC,calc}
\marginsize{3cm}{3cm}{2.5cm}{2.5cm}
\providecommand{\abs}[1]{\lvert#1\rvert}
\providecommand{\norm}[1]{\lVert#1\rVert}
\renewcommand{\thesubsection}{\thesection.\alph{subsection}}
\author{Emanuel Alfredo Cortez Médici}
\title{Trabajo Práctico 1}
\DeclareGraphicsExtensions{.jpeg,.png}
\usepackage{beramono}% monospaced font with bold variant

\usepackage{listings}
\usepackage{color}



\begin{document}
	\begin{titlepage}
	\centering
	
	\includegraphics[width=0.1\textwidth]{fotos/unsl.png} 
	
	{\scshape\LARGE Universidad Nacional de San Luis\par}
	{\scshape Facultad de Ciencias Físico Matemáticas y Naturales.\par}
	{\scshape Ingeniería Electrónica con O.S.D.\par}
	\vspace{1.5cm}
	{\scshape\Large Arquitectura de computadoras \par}
	\vspace{1.5cm}
	{\huge\bfseries Trabajo Práctico N$^{\text{o}}$ 4
	
    Camino de datos y control del microprocesador\par}
	\vspace{2cm}
	Alumno\par
	{\Large\itshape 	
	Cortez Médici Emanuel\par}
	\vfill
	Profesores Responsables\par
	Ing. Andrés~\textsc{Airabella}\\
	Ing. Astrid~\textsc{Andrada}
	
	\vfill

% Bottom of the page
	{\large \today\par}
\end{titlepage}

\newpage

\section{Genere dibujos independientes para cada uno de los bloques que va a utilizar para
construir el procesador.}

Se generan los siguientes bloques independientes para cada uno de los bloques que se necesitan para construir el procesador:


\section{Realice el dibujo de un Datapath completo para el set de instrucciones propuesto.
Indique en el dibujo anchos de todos los buses y nombres de las señales
intermedias que luego utilizará en el código.}

\section{Realice una tabla de verdad para todas las señales de control.}

\section{Cree un repositorio en www.gitlab.com siguiendo este tutorial:
https://alejandrojs.wordpress.com/2017/06/01/como-empezar-a-usar-git-con-gitlab/}

\section{Elabore un informe con el desarrollo de los ejercicios. Este informe deberá
escribirse dentro del mismo repositorio creado en el ejercicio anterior, utilizando el
formato “Markdown”.}


\end{document}