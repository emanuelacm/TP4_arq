\documentclass[11pt,a4paper]{article}
\usepackage{minted}
\usepackage[utf8]{inputenc}
\usepackage[T1]{fontenc}
\usepackage{amsmath}
\usepackage{array, multirow, bigdelim, makecell, booktabs} 
\usepackage{color}
\usepackage{amsfonts}
\usepackage[dvipsnames,table,xcdraw]{xcolor}
\usepackage{amssymb}
\usepackage{caption}
\usepackage{subcaption}
\usepackage{wrapfig}
\usepackage{lscape}
\usepackage{rotating}
\usepackage[spanish]{babel}
\usepackage[section]{placeins}
\usepackage{subcaption}
\usepackage{graphicx}
\usepackage{epstopdf}
\usepackage[font=scriptsize,labelfont=bf]{caption}
\usepackage{anysize}
%\usepackage{tikz}
\usepackage[american]{circuitikz}
\usetikzlibrary{arrows,shapes.gates.logic.US,shapes.gates.logic.IEC,calc,shapes.geometric,quotes,decorations.markings}
\usetikzlibrary{babel}
\marginsize{3cm}{3cm}{2.5cm}{2.5cm}
\providecommand{\abs}[1]{\lvert#1\rvert}
\providecommand{\norm}[1]{\lVert#1\rVert}
\renewcommand{\thesubsection}{\thesection.\alph{subsection}}
\author{Emanuel Alfredo Cortez Médici}
\title{Trabajo Práctico 1}
\DeclareGraphicsExtensions{.jpeg,.png}
\usepackage{beramono}% monospaced font with bold variant

\usepackage{listings}
\usepackage{color}


\begin{document}
	\begin{titlepage}
	\centering
	
	\includegraphics[width=0.1\textwidth]{fotos/unsl.png} 
	
	{\scshape\LARGE Universidad Nacional de San Luis\par}
	{\scshape Facultad de Ciencias Físico Matemáticas y Naturales.\par}
	{\scshape Ingeniería Electrónica con O.S.D.\par}
	\vspace{1.5cm}
	{\scshape\Large Arquitectura de computadoras \par}
	\vspace{1.5cm}
	{\huge\bfseries Trabajo Práctico N$^{\text{o}}$ 4
	
    Camino de datos y control del microprocesador\par}
	\vspace{2cm}
	Alumno\par
	{\Large\itshape 	
	Cortez Médici Emanuel\par}
	\vfill
	Profesores Responsables\par
	Ing. Andrés~\textsc{Airabella}\\
	Ing. Astrid~\textsc{Andrada}
	
	\vfill

% Bottom of the page
	{\large \today\par}
\end{titlepage}

\newpage

\section{Genere dibujos independientes para cada uno de los bloques que va a utilizar para
construir el procesador.}

Se generan los siguientes bloques independientes para cada uno de los bloques que se necesitan para construir el procesador:
\begin{center} 
\resizebox{\textwidth}{!}{%
\begin{tikzpicture}[thick]
	\tikzset{mux/.style={muxdemux, muxdemux def={Lh=4, Rh=2, NL=2, NB=1, NR=1}}}
	\tikzset{ALU ej/.style={ALU, muxdemux def={NL=2, NT=0,NB=0, NR=1}}}

	\tikzset{PC/.style={muxdemux,muxdemux def={Lh=4, NL=2, Rh=4, NR=1,NB=0, w=2, NT=0}}}

	\tikzset{Mem/.style={muxdemux,muxdemux def={Lh=4, NL=3, Rh=4, NR=1,NB=0, w=4, NT=0, square pins=1},no input leads, external pins width=0.4}}

	\tikzset{banco/.style={muxdemux,muxdemux def={Lh=8, NL=6, Rh=8, NR=3,NB=0, w=4, NT=0, square pins=1},no input leads, external pins width=0.4}}


    % Equipos
    \node[and port, fill=gray!50] (and1) at (3,0) {AND};
    \node[or port, fill=gray!50] (or1) at (3,3) {OR};
    \node[mux, fill=gray!50] (mux1) at (3,-3) {Mux};
	\node[ALU ej, fill=gray!50] (alu1) at (9,3) {\rotatebox{90}{ALU}}; 
	\node[adder, fill=gray!50] (sum) at (9,0) {};
	\node[PC, fill=gray!50] (programc) at (9,-3) {PC};
	\node[Mem, fill=gray!50] (mem) at (-6,3) {\begin{tabular}{c} Memoria \\ de \\ Programa \end{tabular}};
	\node[Mem, fill=gray!50] (mem_data) at (-6,0) {\begin{tabular}{c} Memoria \\ de \\ Datos \end{tabular}};
	\node[banco, fill=gray!50] (regis) at (-6,-4) {\begin{tabular}{c} Banco \\ de \\ Registros \end{tabular}};

    % Labels compuerta and
    \draw (and1.in 1) -- ++(-0.75,0) node[left]{$X_1$};
    \draw (and1.in 2) -- ++(-0.75,0) node[left]{$X_2$};
    \draw (and1.out) -- ++(0.75,0) node[right]{$S$};

    % Labels compuerta or
    \draw (or1.in 1) -- ++(-0.75,0) node[left]{$X_1$};
    \draw (or1.in 2) -- ++(-0.75,0) node[left]{$X_2$};
    \draw (or1.out) -- ++(0.75,0) node[right]{$S$};

 	% Labels mux
	\draw (mux1.lpin 1) -- ++(-0.75,0)  node[left]{$X_1$};
	\draw (mux1.lpin 1) ++(0.3,0) node[right, font=\small]{0};
	\draw (mux1.lpin 2) -- ++(-0.75,0) node[left]{$X_2$}; 
	\draw (mux1.lpin 2) ++(0.3,0) node[right, font=\small]{1};
	\draw (mux1.rpin 1) -- ++(0.75,0) node[right]{$Y$};
	\draw (mux1.bpin 1) -- ++(-1.75,-1) node[left]{$Sel$};

	%Label ALU
	\draw (alu1.lpin 1) -- ++(-0.75,0)  node[left]{$A_i$};
	\draw (alu1.lpin 2) -- ++(-0.75,0)  node[left]{$B_i$};
	\draw (alu1.rpin 1) -- ++(0.75,0) node[right,inputarrow]{$Y_o$};

	%Label sumador
	\draw (sum.south) -- ++(0,-0.75)  node[left]{$A_i$};
	\draw (sum.west) -- ++(-0.75,0)  node[left]{$B_i$};
	\draw (sum.east)-- ++(0.75,0)  node[right,inputarrow]{$Y_o$};

	%Label PC
	\draw (programc.lpin 1) -- ++(-0.75,0)  node[left]{$A_i$};
	\draw (programc.lpin 2) -- ++(-0.75,0)  node[left]{$B_i$};
	\draw (programc.rpin 1) -- ++(0.75,0) node[right,inputarrow]{$Y_o$};

	%Label memoria de programa
	\draw (mem.rpin 1) -- (mem.brpin 1);
	\draw (mem.lpin 2) -- (mem.blpin 2);
	\draw (mem.lpin 3) -- (mem.blpin 3);
	\draw (mem.lpin 2) -- ++(-0.75,0) node[left]{$ADDR_i$};
	\draw (mem.lpin 3) -- ++(-0.75,0)  node[left]{$CLK_i$};
	\draw (mem.rpin 1) -- ++(0.75,0) node[right,inputarrow]{$DATA_o$};

	%Label memoria de datos
	\draw (mem_data.rpin 1) -- (mem_data.brpin 1);
	\draw (mem_data.lpin 1) -- (mem_data.blpin 1);
	\draw (mem_data.lpin 2) -- (mem_data.blpin 2);
	\draw (mem_data.lpin 3) -- (mem_data.blpin 3);
	\draw (mem_data.lpin 1) -- ++(-0.75,0) node[left]{$ADDR_i$};
	\draw (mem_data.lpin 2) -- ++(-0.75,0) node[left]{$DATA_i$};
	\draw (mem_data.lpin 3) -- ++(-0.75,0)  node[left]{$CLK_i$};
	\draw (mem_data.rpin 1) -- ++(0.75,0) node[right,inputarrow]{$DATA_o$};

	%Label banco de registros
	\draw (regis.rpin 1) -- (regis.brpin 1);
	\draw (regis.rpin 3) -- (regis.brpin 3);

	\draw (regis.lpin 1) -- (regis.blpin 1);
	\draw (regis.lpin 2) -- (regis.blpin 2);
	\draw (regis.lpin 3) -- (regis.blpin 3);
	\draw (regis.lpin 5) -- (regis.blpin 5);
	\draw (regis.lpin 6) -- (regis.blpin 6);

	\draw (regis.lpin 1) -- ++(-0.75,0) node[left]{$A_i$};
	\draw (regis.lpin 2) -- ++(-0.75,0) node[left]{$B_i$};
	\draw (regis.lpin 3) -- ++(-0.75,0)  node[left]{$C_i$};
	\draw (regis.lpin 5) -- ++(-0.75,0) node[left]{$Wc_i$};
	\draw (regis.lpin 6) -- ++(-0.75,0)  node[left]{$CLK_i$};

	\draw (regis.rpin 1) -- ++(0.75,0) node[right,inputarrow]{$Ra_o$};
	\draw (regis.rpin 3) -- ++(0.75,0) node[right,inputarrow]{$Rb_o$};

\end{tikzpicture}}
\end{center}

\section{Realice el dibujo de un Datapath completo para el set de instrucciones propuesto.
Indique en el dibujo anchos de todos los buses y nombres de las señales
intermedias que luego utilizará en el código.}

Se procede a realizar el Datapath del procesador tilizando la librería \textit{circuitikz}.


\section{Realice una tabla de verdad para todas las señales de control.}

Se realiza una tabla de verdad con todas las señales de control:

% Table generated by Excel2LaTeX from sheet 'Hoja2'
\begin{table}[htbp]
	\centering
	  \begin{tabular}{|c|c|c|c|c|c|}
	  \hline
	  Entrada o salida & Nombre de la señal & R-format & ld & sd & beq \\
	  \hline
	  \multirow{7}{*}{Entrada} & I[6] & 0    & 0    & 0    & 1 \\
  \cline{2-6}         & I[5] & 1    & 0    & 1    & 1 \\
  \cline{2-6}         & I[4] & 1    & 0    & 0    & 0 \\
  \cline{2-6}         & I[3] & 0    & 0    & 0    & 0 \\
  \cline{2-6}         & I[2] & 0    & 0    & 0    & 0 \\
  \cline{2-6}         & I[1] & 1    & 1    & 1    & 1 \\
  \cline{2-6}         & I[0] & 1    & 1    & 1    & 1 \\
	  \hline
	\multirow{8}{*}{Salida} & ALUSrc & 0    & 1    & 1    & 0 \\
  \cline{2-6}         & MemtoReg & 0    & 1    & X & X \\
  \cline{2-6}         & RegWrite & 1    & 1    & 0    & 0 \\
  \cline{2-6}         & MemRead & 0    & 1    & 0    & 0 \\
  \cline{2-6}         & MemWrite & 0    & 0    & 1    & 0 \\
  \cline{2-6}         & Branch & 0    & 0    & 0    & 1 \\
  \cline{2-6}         & ALUOp1 & 1    & 0    & 0    & 0 \\
  \cline{2-6}         & ALUOp0 & 0    & 0    & 0    & 1 \\
	  \hline
	  \end{tabular}%
	  \caption{Tabla de verdad con las entradas y salidas de las señales de control.}
	\label{tab:addlabel}%
  \end{table}%
  



\section{Cree un repositorio en www.gitlab.com siguiendo este tutorial:
\small https://alejandrojs.wordpress.com/2017/06/01/como-empezar-a-usar-git-con-gitlab/}

\section{Elabore un informe con el desarrollo de los ejercicios. Este informe deberá
escribirse dentro del mismo repositorio creado en el ejercicio anterior, utilizando el
formato “Markdown”.}


\end{document}